\documentclass[boxes]{gsypset}

% Info for header
\mailbox{}
\initials{}
\collaborators{}
\class{Math 65}
\assignment{HW 6}
\duedate{May 24, 2016}

\DeclareMathOperator{\spn}{span}

\begin{document}

\textsf{\textbf{Note:}} You need to be able to calculate eigenvalues and eigenvectors by hand on the final exam. If you haven't honed this skill well enough to the point where you can do it quickly and error-free, then perform the calculations on your homework assignments by hand. If you don't need the practice, then feel free to use the computer to do those calculations for you.
\vspace{.5\baselineskip}\hrule

\begin{problem}
	Let
	\[
		A=\bm{2 & 0 & 0 & 1 \\ 0 & 1 & 0 & 0 \\ 0 & 0 & 1 & 0 \\ 1 & 0 & 0 & 2}
	\]
	\begin{subproblems}
		\subproblem Find eigenvalues and eigenvectors of this symmetric matrix. 
			(Review Poole Section 4.2 if you've forgotten how to calculate a determinant using cofactor expansions.)
			
			\textbf{Note:} Eigenvectors corresponding to the same eigenvalue don't have to be orthogonal, 
			but you can choose them so that all four eigenvectors will form an orthogonal basis for $\mathbb{R}^4$.
			\begin{solution}
				
			\end{solution}
		\subproblem Normalize all of your eigenvectors (that is, scale them so they have unit length) 
			and arrange them as columns of a matrix $Q$. Then verify that $Q^TQ=I_4$ and $QQ^T=I_4$.
			\begin{solution}
				
			\end{solution}
		\subproblem Calculate $Q^TAQ$. (Before you calculate it, think about what you expect it to be.)
			\begin{solution}
				
			\end{solution}
		\subproblem Let $\hat{\mathbf{v}}_1,\dots,\hat{\mathbf{v}}_4$ be your orthogonal eigenvectors. 
			Since they are linearly independent, they form an orthogonal basis for $\mathbb{R}^4$. 
			Let $B=\{\hat{\mathbf{v}}_1,\dots,\hat{\mathbf{v}}_4\}$. 
			\textit{Without solving a simultaneous system of four coupled equations for four unknowns}, 
			calculate
			\[
				\bm{1\\2\\3\\4}_B
			\]
			\begin{solution}
				
			\end{solution}
	\end{subproblems}
\end{problem}

\begin{problem}
	Let $W$ be the subspace of $\mathbb{R}^4$ that is spanned by these three vectors:
	\[
		\mathbf{x}_1=\bm{2\\-1\\1\\2},
		\qquad
		\mathbf{x}_2=\bm{3\\-1\\0\\4},
		\qquad
		\mathbf{x}_3=\bm{1\\1\\1\\1}.
	\]
	\begin{subproblems}
		\subproblem Use the Gram-Schmidt Process to calculate $\mathbf{v}_1,\mathbf{v}_2,\mathbf{v}_3$, 
			three vectors that form an orthogonal basis for $W$.
			\begin{solution}
				
			\end{solution}
		\subproblem Normalize $\mathbf{v}_1,\mathbf{v}_2,\mathbf{v}_3$ to get 
			$\{\hat{\mathbf{v}}_1,\hat{\mathbf{v}}_2,\hat{\mathbf{v}}_3\}$, an orthonormal basis for $W$.
			\begin{solution}
				
			\end{solution}
		\subproblem Find a unit vector $\hat{\mathbf{v}}_4$ that allows 
			$B=\{\hat{\mathbf{v}}_1,\dots,\hat{\mathbf{v}}_4\}$ to be an orthonormal basis for $\mathbb{R}^4$.
			
			\textbf{Hint:} One way to do this to construct a matrix matrix $Q$ whose columns are 
			$\hat{\mathbf{v}}_1,\dots,\hat{\mathbf{v}}_4$. 
			Find the missing fourth column of the matrix so that $QQ^T=Q^TQ=I$.
			\begin{solution}
				
			\end{solution}
		\subproblem \textit{Without solving a simultaneous system of four coupled equations for four unknowns},
			calculate
			\[
				\left[\begin{matrix}1\\2\\3\\4\end{matrix}\right]_B
			\]
			\begin{solution}
				
			\end{solution}
	\end{subproblems}
\end{problem}

\begin{problem}
	Follow the method outlined in the lecture notes from Friday, May 20 to find the general solution to
	\[
		\begin{cases}
			x_1'(t)&\hspace{-.12in}=-x_1+2x_2\\
			x_2'(t)&\hspace{-.12in}=2x_1-4x_2.
		\end{cases}
	\]
	Then, find the specific solution corresponding to the initial
	conditions $x_1(0)=1$, $x_2(0)=1$, and to describe what happens to this
	solution as $t\to\infty$.\\
	
	\textbf{Note:} Follow the method in the lecture notes closely. 
	Make sure you understand each step. 
	In subsequent problems, you can skip any steps in the derivation that you have already explained.
\end{problem}
\begin{solution}
	
\end{solution}

\begin{problem}
	Find the general solution to
	\[
		\begin{cases}
			x_1'(t)&\hspace{-.12in}=x_1/2+9x_2\\
			x_2'(t)&\hspace{-.12in}=x_1/2+2x_2.
		\end{cases}
	\]
	Then, find the specific solution corresponding to the initial
	conditions $x_1(0)=1$, $x_2(0)=1$, and to describe what happens to this
	solution as $t\to\infty$.
\end{problem}
\begin{solution}
	
\end{solution}

\begin{problem}
	Find the general solution to
	\[
		\begin{cases}
			x_1'(t)&\hspace{-.12in}=-2x_1-x_2\\
			x_2'(t)&\hspace{-.12in}=x_1-4x_2, 
		\end{cases}
	\]
	Then, find the specific solution corresponding to the initial
	conditions $x_1(0)=7$, $x_2(0)=0$, and to describe what happens to this
	solution as $t\to\infty$.
	
	\textbf{Hint:} This time, you will find that the matrix cannot be diagonalized. 
	However, the eigenvalue $\lambda$ that you obtain is still useful. 
	Let $x_1(t)=u_1(t)e^{\lambda t}$ and $x_2(t)=u_2(t)e^{\lambda t}$ and 
	plug these into the two DEs to get two new DEs for $u_1(t)$ and $u_2(t)$ that are easier to solve.
\end{problem}
\begin{solution}
	
\end{solution}
\end{document}

