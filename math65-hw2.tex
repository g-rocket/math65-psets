\documentclass[oneperpage]{gsypset}

\name{}
\class{Math 65}
\assignment{HW 2}
\duedate{May 18, 2016}

\DeclareMathOperator{\rank}{rank}

\begin{document}
	\begin{problem}[6.2.20]
		Determine whether the set $\mathcal{B}$ is a basis for the vector space $V$.
		\begin{align*}
			V &= M_{22} \\
			\mathcal{B} &= \left\{
					\bm{1&0\\0&1},
					\bm{0&1\\1&0},
					\bm{1&1\\0&1},
					\bm{1&0\\1&1}
				\right\}
		\end{align*}
	\end{problem}
	\begin{solution}
		
	\end{solution}
	
	\begin{problem}[6.2.25]
		Determine whether the set $\mathcal{B}$ is a basis for the vector space $V$.
		\begin{align*}
			V &= \mathscr{P}_2 \\
			\mathcal{B} &= \left\{
					x,
					1+x,
					x-x^2
				\right\}
		\end{align*}
	\end{problem}
	\begin{solution}
		
	\end{solution}
	
	\begin{problem}[6.2.28]
		Find the coordinate vector of $p(x) = 1 + 2x + 3x^2$ with respect to the basis 
		$\mathcal{B}=\{l+x,1-x,x^2\}$ of $\mathscr{P}_2$.
	\end{problem}
	\begin{solution}
		
	\end{solution}
	
	\begin{problem}[6.2.38]
		Find the dimension of the vector space $V$ and give a basis for $V$.
		\[
			V = \{A\ \text{in}\ M_{22} \colon A\ \text{is skew-symmetric}\}
		\]
	\end{problem}
	\begin{solution}
		
	\end{solution}
	
	\begin{problem}[6.3.7]
		With
		\begin{align*}
			p(x) &= 1 + x^2 \\
			\mathcal{B} &= \{1+x, 1-x\} \\
			\mathcal{C} &= \{1,x,x^2\}
		\end{align*}
		in $\mathscr{P}_2$,
		\begin{subproblems}[(a)]
			\subproblem
				Find the coordinate vectors $[\mathbf{x}]_\mathcal{B}$ and $[\mathbf{x}]_\mathcal{C}$
				of $\mathbf{x}$ with respect to the bases $\mathcal{B}$ and $\mathcal{C}$ respectively.
				\begin{solution}
					
				\end{solution}
				
			\subproblem
				Find the change-of-basis matrix $P_{C \leftarrow B}$ from $\mathcal{B}$ to $\mathcal{C}$.
				\begin{solution}
					
				\end{solution}
				
			\subproblem
				Use your answer from part (b) to compute $[\mathbf{x}]_\mathcal{C}$, 
				and compare you answer with the one found in part (a).
				\begin{solution}
					
				\end{solution}
				
			\subproblem
				Find the change-of-basis matrix $P_{B \leftarrow C}$ from $\mathcal{C}$ to $\mathcal{B}$.
				\begin{solution}
					
				\end{solution}
				
			\subproblem
				Use your answers to parts (c) and (d) to compute $[\mathbf{x}]_\mathcal{C}$, 
				and compare you answer with the one found in part (a).
				\begin{solution}
					
				\end{solution}
		\end{subproblems}
	\end{problem}
	
	\begin{problem}[6.3.18]
		Express $p(x)=1 + 2x - 5x^2$ as a Taylor polynomial about $a = -2$
	\end{problem}
	\begin{solution}
		
	\end{solution}
	
	\begin{problem}[6.3.21]
		Let $\mathcal{B}, \mathcal{C}, \mathcal{D}$ be bases for a finite-dimensional vector space $V$.
		Prove that
		\[
			P_{\mathcal{D} \leftarrow \mathcal{C}} P_{\mathcal{C} \leftarrow \mathcal{B}}
				= P_{\mathcal{D} \leftarrow \mathcal{B}}
		\]
	\end{problem}
	\begin{solution}
		
	\end{solution}
	
	\begin{problem*}[Additional problem \#1]
		For each square matrix below, calculate its eigenvalues and eigenvectors.
		Then verify that $PDP^{-1}$ is equal to the original matrix, 
		where $D$ is a diagonal matrix with your eigenvalues along its diagonal
		and $P$ is a matrix with your eigenvectors as its columns.
		\begin{subproblems}[(a)]
			\subproblem $\bm{1&0\\0&0}$
			\begin{solution}
				
			\end{solution}
			
			\subproblem $\bm{0 & -13 & -4 \\ 0 & -3 & 0 \\ 1 & 13 & 0}$
			\begin{solution}
				
			\end{solution}
		\end{subproblems}
	\end{problem*}
\end{document}