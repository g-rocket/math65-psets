\documentclass[boxes]{gsypset}

% Info for header
\mailbox{}
%\initials{}
\class{Math 65}
\assignment{HW 3}
\duedate{May 19}

\DeclareMathOperator{\spn}{span}

\begin{document}

\problemlist{6.1.47, 6.3.16, 6.4.\{4, 21\}, 6.5.12, 6.6.2, 4.4.22, EC: 6.2.33}

\noindent
\emph{Intructor's Note:} I recommend that you also look at the Chapter Review on
pages 527-528 of Poole and skim the problems to see if there are any concepts or
problems that seem challenging to you. Try some of these problems for more
practice.

\begin{problem}[6.1.47]
	Let $V$ be a vector space with subspaces $U$ and $W$. Give an
	example with $V=\mathbb{R}^2$ to show that $U \cup W$ need not be a
	subspace of $V$.
\end{problem}
\begin{solution}
	
\end{solution}

\begin{problem}[6.3.16]
	Let $\mathcal{B}$ and $\mathcal{C}$ be bases for $\mathscr{P}_2$. 
	If $\mathcal{B}=\{x,1+x,1-x+x^2\}$ and the change-of-basis matrix from 
	$\mathcal{B}$ to $\mathcal{C}$ is
	\[
		P_{\mathcal{C}\leftarrow\mathcal{B}}=\bm{1&0&0\\0&2&1\\-1&1&1},
	\]
	find $\mathcal{C}$.
\end{problem}
\begin{solution}
	
\end{solution}

\begin{problem}[6.4.4]
	Determine whether $T:M_{nn}\to M_{nn}$, defined by $T(A)=AB-BA$,
	where $B$ is a fixed $n\times n$ matrix,
	is a linear transformation.
\end{problem}
\begin{solution}
	
\end{solution}

\begin{problem}[6.4.21]
	\fbox{\begin{minipage}{\linewidth}
		THEOREM 6.14:
		
		Let $T: U \to V$ be a linear transformation. Then,
		\begin{enumerate}[a.]
			\item $T(\mathbf{0}) = \mathbf{0}$.
			\item $T(-\mathbf{v}) = -T(\mathbf{v})$ for all $\mathbf{v}$ in $V$.
			\item $T(\mathbf{u} - \mathbf{v}) = T(\mathbf{u}) - T(\mathbf{v})$
				for all $\mathbf{u}$ and $\mathbf{v}$ in $V$.
		\end{enumerate}
	\end{minipage}}
	
	Prove Theorem 6.14(b).
\end{problem}
\begin{solution}
	
\end{solution}

\begin{problem}[6.5.12]
	$T:M_{22}\to M_{22}$ defined by $T(A)=AB-BA$, where $B=\bm{1&1\\0&1}$
	
	Find either the nullity or the rank of $T$ and
	then use the Rank Theorem to find the other.
\end{problem}
\begin{solution}
	
\end{solution}

\begin{problem}[6.6.2]
	\fbox{\begin{minipage}{\linewidth}
		THEOREM 6.26:
		
		Let $V$ and $W$ be two finite-dimensional vector spaces with bases 
		$\mathcal{B}$ and $\mathcal{C}$ respectively, where 
		$\mathcal{B} = \{\mathbf{v}_1,\dots,\mathbf{v}_n\}$.
		If $T: V \to W$ is a linear transformation, then the $m \times n$ matrix $A$ defined by
		\[
			A = \bm{\left[T(\mathbf{v}_1)\right]_\mathcal{C} &
			        \left[T(\mathbf{v}_2)\right]_\mathcal{C} &
			        \cdots &
			        \left[T(\mathbf{v}_n)\right]_\mathcal{C}}
		\]
		satisfies
		\[
			A[\mathbf{v}]_\mathcal{B} = [T(\mathbf{v})]_\mathcal{C}
		\]
		for every vector $\mathbf{v}$ in $V$.
	\end{minipage}}
	
	Find the matrix $[T]_{\mathcal{C}\leftarrow\mathcal{B}}$
	of the linear transformation $T:V\to W$ with respect to the bases 
	$\mathcal{B}$ and $\mathcal{C}$ of $V$ and $W$, respectively. 
	Verify Theorem 6.26 for the vector \textbf{v} by 
	computing $T(\textbf{v})$ directly and using the theorem.
	
	$T: \mathscr{P}_1 \to \mathscr{P}_1$ defined by
	\begin{align*}
		T(a+bx) &= b-ax, \\
		\mathcal{B} &= \{1+x,1-x\}, \\
		\mathcal{C} &= \{1,x\}, \\
		\textbf{v} &= p(x)=4+2x
	\end{align*}
\end{problem}
\begin{solution}
	
\end{solution}

\begin{problem}[4.4.22]
	Use the method of Example 4.29 to compute
	\[
		\bm{2&0&1\\1&1&1\\1&0&2}^k
	\]
	(assume that $k$ is a positive integer)
	
	[\textit{Hint}: Example 4.29 uses the equality $M^n = PD^nP^{-1}$.]
\end{problem}
\begin{solution}
	
\end{solution}

\begin{problem}[Extra Credit: 6.2.33]
	Let $\{\textbf{u}_1,...,\textbf{u}_m\}$ be 
	a set of vectors in an $n$-dimensional vector space $V$ and let 
	$\mathcal{B}$ be a basis for $V$. 
	Let $S=\{[\textbf{u}_1]_{\mathcal{B}},...,[\textbf{u}_m]_{\mathcal{B}}\}$
	be on the set of coordinate vectors of
	$\{\textbf{u}_1,...,\textbf{u}_m\}$ with respect to $\mathcal{B}$. 
	Prove that $\spn(\textbf{u}_1,...,\textbf{u}_m)=V$ if and only if $\spn(S)=\mathbb{R}^n$.
	
	(Remember that to prove an if-and-only-if theorem, you need to prove both directions.)
\end{problem}
\begin{solution}
	
\end{solution}
\end{document}
