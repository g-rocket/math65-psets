\documentclass[boxes]{gsypset}

% Info for header
\mailbox{}
\initials{}
\collaborators{}
\class{Math 65}
\assignment{HW 12}
\duedate{June 2, 2016}

\usepackage{hyperref}

\begin{document}
	\begin{problem}
		Consider the nonlinear system of DEs
		\begin{align*}
			x' &= x - y + x^2 - xy \\
			y' &= -y + x^2
		\end{align*}
		
		\begin{subproblems}
			\subproblem 
				Find the nullclines of this system. 
				Use the nullclines to perform ``sign analysis.'' 
				In other words, determine the signs of $x'$ and $y'$ 
				in each of the regions of the phase plane delineated by the nullclines.
				\begin{solution}
					
				\end{solution}
			\subproblem Locate the equilibrium points of this system of differential equations.
				\begin{solution}
					
				\end{solution}
			\subproblem For each equilibrium point,
				\begin{itemize}
					\item write down the linearized system of DEs about that point,
					\item calculate the eigenvalues of the linearized system,
					\item 
						use the eigenvalues to determine the behavior 
						(unstable/stable, node/spiral, etc.) of the linearized system about that point,
					\item 
						if possible use that information about the linearized system to infer 
						something about the nature of the equilibrium point 
						of the original nonlinear system of DEs.
				\end{itemize}
				\begin{solution}
					
				\end{solution}
			\subproblem 
				Use the information about these equilibrium points and the sign analysis to 
				sketch orbits by hand on your phase plane.
				\begin{solution}
					
				\end{solution}
			\subproblem 
				Use mathematical software to plot a phase plane portrait of the system and 
				compare it to your hand sketch. 
				Please do this only after you've tried your best to do what you can by hand. 
				Feel free to use the computer-generated picture to correct any details 
				that you might have missed.
				\begin{solution}
					
				\end{solution}
		\end{subproblems}
	\end{problem}
	
	\begin{problem}
		The nonlinear stability of an equilibrium point cannot be
	  reliably predicted if all that is known is that the linearized
	  system has a center there.  This can be demonstrated by these two
	  systems of equations.
		\begin{align*}{3}
			\dot{x} &= y+x(x^2+y^2) & 
				\qquad\qquad &\raisebox{-.1in}[0mm][0mm]{\text{and}} \qquad\qquad &
				\dot{x}&=y-x(x^2+y^2) \\
			\dot{y} &= -x+y(x^2+y^2) &&& \dot{y} &= -x-y(x^2+y^2)
		\end{align*}
		\begin{subproblems}
			\subproblem 
				Linearize both systems about the origin (the only equilibrium point). 
				What type of equilibrium point does the linearized system have at the origin?
				\begin{solution}
					
				\end{solution}
			\subproblem 
				Use computer software to show what the solution trajectories actually look like 
				if you choose initial conditions that are close to the origin. 
				Based on these calculations, what is the true nature of 
				the equilibrium point at the origin for each system of equations?
				\begin{solution}
					
				\end{solution}
			\subproblem (Extra credit) 
				Convert both systems of equations from Cartesian coordinates $(x,y)$ 
				to polar coordinates $(r,\theta)$, where $r^2=x^2+y^2$, etc. 
				What do the converted equations tell you about the stability of the origin in each case?
				\begin{solution}
					
				\end{solution}
		\end{subproblems}
	\end{problem}
	
	%%%%%%%%%%%%%%%%%%%%%%%%%%%%%%%%%%%%%%%%%%%%%
	\begin{problem}
		On the last assignment, you began analyzing the competitive species population model:
		\begin{align*}
			x'(t) &= x(r_1-a_1x-b_1y) \\
			y'(t) &= y(r_2-a_2y-b_2x)
		\end{align*}
		All of the parameters in the DE above ($r_1$, $a_1$, $b_1$, $r_2$, $a_2$, $b_2$) are positive, 
		and we restrict our attention to $x\geq 0$ and $y\geq 0$ only.  
		Depending on these six constants, there are four different 
		(nondegenerate) cases to be considered.
		
		\footnotesize
		\begin{tabular}{c|c|c}
			Case & Condition & Behavior \\ \hline
			1 & 
			$\frac{r_1}{a_1}<\frac{r_2}{b_2}$ \&
			$\frac{r_1}{b_1}<\frac{r_2}{a_2}$
			&
			$x$ goes extinct for any positive initial population values \\
			2 & 
			$\frac{r_1}{a_1}>\frac{r_2}{b_2}$ \&
			$\frac{r_1}{b_1}>\frac{r_2}{a_2}$
			&
			$y$ goes extinct for any positive initial population values \\
			3 & 
			$\frac{r_1}{a_1}<\frac{r_2}{b_2}$ \&
			$\frac{r_1}{b_1}>\frac{r_2}{a_2}$
			&
			both $x,y$ approach coexistence for any positive initial population values \\
			4 & 
			$\frac{r_1}{a_1}>\frac{r_2}{b_2}$ \&
			$\frac{r_1}{b_1}<\frac{r_2}{a_2}$
			&
			one species will go extinct, depending on initial population values
		\end{tabular}
		\normalsize
		
		Please refer to the solutions for Homework \#11 for complete details.
		\begin{subproblems}
			\subproblem 
				In all four cases, there are three extinction equilibrium points: 
				$(0,0)$, $(r_1/a_1,0)$, and $(0,r_2/a_2)$. 
				Use local stability analysis to determine the nature of 
				each equilibrium point in all four cases.
				\begin{solution}
					
				\end{solution}
			\subproblem 
				In cases 3 and 4, there is an additional coexistence equilibrium point at
				\[
					\mathbf{x}_{\text{eq}}
						= \left(x_{\text{eq}},y_{\text{eq}}\right)
						= \left(\frac{r_1a_2-b_1r_2}{a_1a_2-b_1b_2},\frac{a_1r_1-r_2b_2}{a_1a_2-b_1b_2}\right)
				\]
				that is viable. 
				Pick some values for the six constants so that you are in case 3. 
				Use those values to perform local stability analysis to 
				determine the nature of this equilibrium point. 
				Repeat for case 4.
				(We are using specific values for the constants to capture the behavior 
				in cases 3 and 4 since the algebra is a little more intense if you don't do that.)
				\begin{solution}
					
				\end{solution}
			\subproblem 
				Synthesize all of the calculations that you've performed to explain 
				why the populations behave the way they do in all four cases.
				\begin{solution}
					
				\end{solution}
			\subproblem (Extra credit) 
				Perform local stability analysis for arbitrary values of the six constants in cases 3 and 4. 
				\textbf{Hint:} The algebra is not as bad as you think if you are strategic about it.
				\begin{solution}
					
				\end{solution}
		\end{subproblems}
	\end{problem}
	
	%%%%%%%%%%%%%%%%%%%%%%%%%%%%%%%%%%%%%%%%%%%% 
	
	\begin{problem}
		The \textit{Brusselator} system of equations,
		\begin{align*}
			\dot{x} &= a+x^2y-(1+b)x \\
			\dot{y} &= bx-x^2y,
		\end{align*}
		is a mathematical model of a certain chemical reaction.  
		(Google ``Brusselator'' and ``Belousov Zhabotinsky reaction'' to find out more.) 
		The parameters $a$ and $b$ are related to reaction rates, and are therefore positive numbers. 
		These equations have the property that certain choices of parameters lead to 
		long-term oscillatory solutions.
		\begin{subproblems}
			\subproblem Where is the only equilibrium point of this system?
				\begin{solution}
					
				\end{solution}
			\subproblem 
				Assuming that $a=1$, use local stability analysis to 
				describe how the stability of this equilibrium point depends on $b$. 
				In particular, what happens when $b=2$? 
				Remember not to draw any conclusions about the nonlinear differential equation system 
				that are not warranted by the results of local stability analysis.
				\begin{solution}
					
				\end{solution}
			\subproblem (Extra credit) 
				Use a computer to generate some phase portraits to verify your work 
				and to investigate what happens.  
				You will see an attracting ``limit cycle'' (long-term oscillatory solution) 
				appear around the equilibrium point as $b$ increases past 2.
				\begin{solution}
					
				\end{solution}
		\end{subproblems}
	\end{problem}
	
	%%%%%%%%%%%%%%%%%%%%%%%%%%%
	\begin{problem}
		Last June, the Deep Space Climate Observatory, which performs real-time solar wind monitoring 
		(important for warning about geomagnetic storms that can 
		disrupt power grids, telecommunications, etc.) reached its final destination at ``L1," 
		nearly a million miles away from the earth.  
		L1 is one of five equilibrium points, called Lagrange points, 
		that exist in a three-body system consisting of two large bodies 
		(such as the sun and the earth) and one small body (such as a satellite).
		
		In this problem you will determine the stability of L1 in a simplified case 
		where we ignore the fact that the large bodies are orbiting each other. 
		Consider the sun (mass $M_S$) and earth (mass $M_E$) to be a fixed distance $d$ apart. 
		Let $x(t)$ be the distance of the satellite (with mass $m$) from the sun 
		on the line joining the sun and earth.
		
		\textbf{Note \#1:} 
			You don't have to look up the values of the masses of the sun and earth in this problem. 
			You can keep them as variables, along with the gravitational constant. 
			Don't be afraid of a little algebra. 
			Be strategic about how you simplify things.
		
		\textbf{Note \#2:} 
			Google ``Lagrange points'' if you'd like to know more or 
			if you'd like to see the mathematics required 
			if the rotating frame of reference is taken into account. 
			You can read more about the DSCO at \url{http://www.nesdis.noaa.gov/DSCOVR/}.
		
		\begin{subproblems}
			\subproblem 
				Using Newton's Second Law, write the governing differential equation for 
				the motion of the satellite along this line. 
				(Look up the equation for the gravitational force between two objects if you need to.) 
				You may assume that the satellite is positioned in between the sun and earth. 
				Write this second-order ODE as a system of first order equations. 
				Find the equilibrium position of the satellite that is between the sun and earth.
				\begin{solution}
					
				\end{solution}
			\subproblem 
				Linearize your differential equations about the equilibrium point and 
				assess its stability using local stability analysis.
				\begin{solution}
					
				\end{solution}
		\end{subproblems}
	\end{problem}
\end{document}

